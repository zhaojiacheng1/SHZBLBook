% !Tex root = ../../SHZBLBook.tex
% 此处为chapter文件
\chapter{辨太阳病脉证并治\hspace{0.5em}上}

国学者,吾国已往之一种学问。包含中国学术性质与变迁,而并非为与外国绝对不同之学问也。吾国汉代古谚日,少所见,多所怪,见橐驼言马肿背。吾国旧时视外人来华者,不知其学。较进,则知可学其一二端。更进,则知其自有其学术,而与吾国为截然不同。然由今之所见,则知中国之与外国,实为大同小异者也。古代各部落,有知造舟者,有知制车者,各有所能,各有所不知。今外国自工业革命以来,文明日启,距今亦为时不远。由将来观之,东西两洋之文化,犹古代各部落间文化之关系也。又常有以精神文明、物质文明等以区别东西洋之文化。实亦不然。今世之各社会,皆为文明之社会,其程度相差无几,善亦同善,恶亦同恶,固无何高下也。