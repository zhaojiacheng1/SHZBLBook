% !Tex root = ../../SHZBLBook.tex
% 此处为chapter文件
\chapter{辨太阳病脉证并治\hspace{0.5em}中}

中国学术之源渊

\hspace{2em}古代之宗教哲学

\hspace{2em}政治机关经验所得,所谓王官之学

合此两者而生先秦诸子之学,诸家并立

儒家之学独盛

儒家中烦琐之考证,激起空谈原理之反动,偏重《易经》,与道家之学相合,是为魏晋玄学

以上为中国学术自己的发展

至此而佛学输入,为中国所接受。萌芽于汉魏,盛于南北朝,而极于隋唐,其发达之次序,则从小乘至大乘,是为佛学时代,而玄学仍点缀期间。

至唐而反动渐起。至宋而形成理学。理学之性质,可谓摄取佛学之长,而去其不适宜于中国者。

至此中国学术受印度影响之时代,至明亡而衰。

而欧洲学术,适于此时开始输入。近百年来,对中国学术逐渐发生影响。\footnote{【前此与欧洲之接触,仅为技术上,而非学术上,故未受若何之影响。】}

